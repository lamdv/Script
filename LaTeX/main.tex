\typeout{IJCAI-11 Instructions for Authors}

% These are the instructions for authors for IJCAI-11.
% They are the same as the ones for IJCAI-07 with superficical wording
%   changes only.

\documentclass{article}
% The file ijcai11.sty is the style file for IJCAI-11 (same as ijcai07.sty).
\usepackage{ijcai11}

% Use the postscript times font!
\usepackage{times}

% \usepackage[style=authoryear,sorting=ynt]{biblatex}
% \usepackage{bibtex}
% the following package is optional:
%\usepackage{latexsym} 

% Following comment is from ijcai97-submit.tex:
% The preparation of these files was supported by Schlumberger Palo Alto
% Research, AT\&T Bell Laboratories, and Morgan Kaufmann Publishers.
% Shirley Jowell, of Morgan Kaufmann Publishers, and Peter F.
% Patel-Schneider, of AT\&T Bell Laboratories collaborated on their
% preparation.

% These instructions can be modified and used in other conferences as long
% as credit to the authors and supporting agencies is retained, this notice
% is not changed, and further modification or reuse is not restricted.
% Neither Shirley Jowell nor Peter F. Patel-Schneider can be listed as
% contacts for providing assistance without their prior permission.

% To use for other conferences, change references to files and the
% conference appropriate and use other authors, contacts, publishers, and
% organizations.
% Also change the deadline and address for returning papers and the length and
% page charge instructions.
% Put where the files are available in the appropriate places.


\title{Machine Learning for Polygenic risk score}
% \thanks{These match the formatting instructions of IJCAI-07. The support of IJCAI, Inc. is acknowledged.}}
\author{Vu-Lam DANG\\ 
Universite Grenoble Alpes\\
Grenoble, France \\
vu-lam.dang@etu.univ-grenoble-alpes.fr \\
\\
Supervised by: Dr. Michael G. B. Blum.} % Mosig student

\begin{document}




\maketitle

{% Mosig student
  {\hbox to0pt{\vbox{\baselineskip=10dd\hrule\hbox
to\hsize{\vrule\kern3pt\vbox{\kern3pt
\hbox{{\small I understand what plagiarism entails and I declare that this report }}
\hbox{{\small is my own, original work. }}
\hbox{{\small Vu-Lam DANG, 26/04/2019:}}
\kern3pt
}\hfil%\kern3pt
\vrule
}\hrule}
}}
}


\begin{abstract}
The advent of genomic prediction as a viable 
diagnostic tool for certain disease have become 
a reality in the last decade. One particular 
interesting metric is called Polygenic Risk Score, 
which sumaries the genetic. For this internship project, 
we interested in applying advancement in the field of 
machine learning to improve the predictive power of PRS.
\end{abstract}

\section{Introduction}

\subsection{Polygenic Risk Score}
Polygenic Risk Score (PRS) \cite{Dudbridge2013} is a single perimeter metric
constructed from the weighted sum of associated alleles within each subject.
Given a pair of traits $Y = (Y 1 , Y 2)'$ 0 expressed as a weighted sum of m genetic
effects and a bias indicate environmental and unaccounted genetic effects:

$$Y = \beta' G + \epsilon = [\sum_{i=1}^{m} \beta_{i1} G_1 + \epsilon_1, \sum_{i=1}^{m} \beta_{i2} G_2 + \epsilon_2]$$

where $\beta$ is a $m*2$ weight matrix, $G$ is $m$ length vector of genetic markers 
and $\epsilon$ is a pair of random error that's independent from G.

The Polygenic score is defined as:

$$\hat{S} = \sum_{i=1}^{m} \beta_{i1} G_1$$
Association between a trait and its composite score highly implies there exist a
genetic signal among the markers, and the evident of genetic effect when there is
no obvious candidate can be obtained. Currently polygenic score have been used
for association testing rather than predicting complex traits \cite{Dudbridge2013}.

Howerver, \cite{Inouye1883} conducted a PRS study on 1.7 million genetic
variants using UK Biobank database on Coronary Atery Disease, and found
strong association between PRS and hazard ration for CAD; demonstrated
the power of genomic risk prediction to stratify individuals, and highlights the
possibility for genomic screening early in life to support risk predition and
preventive treatement.

\subsection{Regression Method}


\subsection{Objective}

\section{Proposal}

For this work we would like to pursuit 2 different direction. The first direction is 
called stacking, in which one model with different values for input parameters, 
or several different models is combined to create more better prediction.

The second direction is to create a sparse model that closely resemble the predictive
power of genome-wide risk score.

\section{Methodology}



\section{Conclusion}

Polygenic Score have been widely studied as a good tool to identify and early
diagnose high risked individual. In order to make it a viable tool for clinical usage,
improvement have to be make to decrease false negative and reduce computational
complexity of the model. By incorporate machine learning technique, we hope to
eventually develop PRS into a powerful and accurate metric and tool to assess
and identify clinical risk factor.

%% The file named.bst is a bibliography style file for BibTeX 0.99c
% \bibliographystyle{named}
\bibliography{main}

\end{document}

